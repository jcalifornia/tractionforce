\RequirePackage{lineno}
\documentclass[aps,prl,reprint,twocolumn,groupedaddress,showpacs]{revtex4-1}
%\documentclass[aps,prl,reprint,twocolumn,groupedaddress,showpacs]{revtex4}
%\documentclass[10pt,reprint]{iopart}
%\documentclass[11pt]{article}
%\documentclass{pnastwo}
%\usepackage{color}
\usepackage[pdftex]{graphicx}
\usepackage{epstopdf}
\usepackage{bm}
\usepackage{amssymb}
\usepackage{psfrag}
\usepackage{color}
\usepackage{amsbsy}      
\usepackage{amsmath}
\usepackage{lineno}
\usepackage{natbib} 
\bibliographystyle{unsrt}
\usepackage{amsmath}
\graphicspath{{./figures/}}
%\usepackage{MnSymbol}



\def\alphaeff{\alpha_{\mathrm{eff}}}
\def\a{\alpha}
\def\b{{\bf b}}
\def\c{{\bf c}}
\def\d{{\bf d}}
\def\dd{\mbox{d}}
\def\dz{\delta z}
\def\ve{\varepsilon}
\def\eps{\epsilon}
\def\f{{\bf f}}
\def\g{\gamma}
\def\r{{\bf r}}
\def\rp{{\bf r}_{\perp}}
\def\q{{\bf q}}
\def\k{\kappa}
\def\u{{\bf u}}
\def\t{{\bf t}}
\def\w{{\bf w}}
\def\x{{\bf x}}
\def\y{{\bf y}}
\def\l{\ell}
\def\o{\omega}
\def\p{{\bf p}}
\def\s{\sigma}
\def\vp{\varphi}
\def\D{\Delta}
\def\F{{\bf F}}
\def\H{{\bf H}}
\def\K{{\bf K}}
\def\L{{\bf L}}
\def\Q{{\bf Q}}
\def\S{{\bf S}}
\def\T{{\bf T}}
\def\U{{\bf U}}
\def\V{{\bf V}}
\def\W{{\bf W}}  
\def\F{{\bf F}}                
\def\P{{\bf P}}
\def\Pt{\tilde{P}}
%\def\s{\mathbf{\sigma}}
\newcommand{\bs}{\boldsymbol{\sigma}}
\newcommand{\Conv}{\mathop{\scalebox{1.5}{\raisebox{-0.2ex}{$\ast$}}}}%

%\usepackage{fullpage}
\newcommand{\RR}{\mathbb{R}}


\begin{document}

\title{Reconstruction of localized force distributions in cells and tissues from substrate displacements using physically-consistent regularization}

%\author{Yanli Liu and Tom Chou} 
\affiliation{Depts. of
Biomathematics and Mathematics, UCLA, Los Angeles, CA 90095-1766}


%\maketitle 

%\begin{article}


%\runninglinenumbers*

\begin{abstract}
We develop a method to reconstruct, from measured displacements of the
underlying elastic substrate, the spatially dependent forces that
cells or tissues impart on them. Since these sources of force
typically arise from focal adhesions, with are localized or
``compact,'' and discontinuous, we solve this inverse problem using
methods of optimization useful for image segmentation. In addition to
the standard quadratic data mismatch terms (that defines least-squares
fitting), we motivate a term in in the objective function which
penalizes variations in the tensor invariants of the reconstructed stress while
preserving boundaries.   By minimizing
the objective function subject to physical constraints, we are able to
efficiently reconstruct stress fields with localized structure from
simulated and experimental substrate displacements. We provide a
numerical method for setting up a discretized inverse problem that that
 is solvable by standard convex optimization techniques. Our method incorporates
the exact solution of the forward problem accurate to first-order finite-difference approximation
in the stress tensor. For utility with newly-available high-resolution data, we motivate
the use of distance-based cutoffs for data inclusion and find under loose regularity
conditions the reconstruction error that results.
\end{abstract}
\maketitle

\section{Introduction}

Cell motility and response to signals have hitherto almost always been
studied in two-dimensional geometries in which cells are placed on a
flat elastic substrate.  Dynamic adhesion between the cells and the
substrate are realized through {\it e.g.}, lamellapodia, filapodia,
and dynamically reorganizing focal adhesions.  Such structures are
spatially localized, as shown in Fig.~\ref{FIG1}. Similarly, on larger
length scales, a collection of cells can give rise to localized stress
distributions. For example, the leading edge of a cell layer produces
the pulling force that leads to migration in wound healing assays.

\begin{figure}[t]
\begin{center}
\includegraphics[width=\linewidth]{Fig1}
\caption{A schematic of an isolated cell. (a) The boundary of the cell
  footprint is denoted by the dashed curve, the stress field is
  represented by the red regions that impart a stress ${\bf F}(x,y)$
  on the surface. Displacements ${\bf u}(\r_{i})$ of the elastic
  medium are measured at position $\r_{i} =
  x_{i}\hat{x}+y_{i}\hat{y}+z_{i}\hat{z}$ (blue dots) that can be
  inside or outside the cell footprint, on the surface ($z_{i}=0$), or
  below the surface ($z_{i}<0$). (b). A perspective view of the
  elastic substrate and cellular footprint.}
\label{FIG1}
\end{center}
\end{figure}


Dynamically varying force generating structures are often small and
difficult to image, especially without biochemical modification such
as incorporation of fluorescent dyes. Therefore, other methods for
inferring their positions and magnitudes have been developed. The
simplest method relies on measuring the displacement of fiduciary
markers, such as gold nanoparticles, embedded in the elastic
substrate. The measured displacements are an indirect probe of the
force-generating structures.  Any inversion method should be able to
not only reconstruct the positions and magnitudes of the stress field,
but should ideally be able to capture potentially sharp boundaries of
the stress-generating structures.

Therefore, we develop a novel method for elastic stress source
recovery using ideas developed for image segmentation.  This class of
methods relies on optimization that uses an $L_{1}$ regularization
term in the objective function.  This type of regularization term is
not derived from a fundamental physical law, but represents a prior
knowledge that the function to be recovered is sparse in content
except near edges.  Nonetheless, our new objective will be constructed
to obey physical constraints and symmetries.

In the next section, we review the basic linear equations of
elasticity that describe the displacement field as a function of an
arbitrary surface stress distribution. This model is then used to
construct the data mismatch term in an objective function. We then
motivate regularization and constraint terms to construct the full
objective function. Finally, given simulated and experimental data, we
minimize our objective function using a modified split-Bregman
algorithm and reconstruct the given stress fields. Our method provides
good reconstruction of localized structures that exhibit desirable
qualities such as the suppression of Gibbs ringing phenomenon at the
boundaries of the stress structures.


\section{Elastic Model}

In this section, we explicitly describe the elastic green's function
associated with a point force applied to the surface of a
semi-infinite half-space, as shown in Fig.~\ref{FIG1}(b). The domain
of the elastic medium is ${\cal D}=\left\{(x,y,z)|x,y\in
R,z\leq0\right\}$. We assume that the elastic medium is infinite
in depth ($d\to \infty$) and lateral extent.

The Green's function for linear isotropic elasticity theory 
in the half-space is given by

\begin{align}
\lefteqn{G^0_{ss}(x,y,z) = }\nonumber\\
&=\frac{1+\nu}{2\pi E}\left[\frac{2(1-\nu)R_{\perp}-z}{R_{\perp}(R_{\perp}-z)} + 
\frac{[2R_{\perp}(\nu R_{\perp}-z)+z^{2}]s^{2}}{R_{\perp}^{3}(R_{\perp}-z)^{2}}\right],
\end{align}
%\begin{equation}
%G_{yy} =\frac{1+\nu}{2\pi E}\left[\frac{2(1-\nu)R_{\perp}-z}{R_{\perp}(R_{\perp}-z)}
%+\frac{[2R_{\perp}(\nu R_{\perp}-z)+z^{2}]y^{2}}{R_{\perp}^{3}(R_{\perp}-z)^{2}}\right],
%\end{equation}
\begin{equation}
G^0_{zz}(x,y,z) =\frac{1+\nu}{2\pi E}\left(\frac{2(1-\nu)}{R_{\perp}}+\frac{z^{2}}{R_{\perp}^{3}}\right),
\end{equation}
\begin{equation} 
G^0_{xy}(x,y,z) = G_{yx}=\frac{1+\nu}{2\pi E}\frac{[2R_{\perp}(\nu R_{\perp}-z)+z^{2}]xy}{R_{\perp}^{3}
(R_{\perp}-z)^{2}},
\end{equation}
\begin{equation}
G^0_{sz, zs}(x,y,z) =\frac{1+\nu}{2\pi E}\left(\frac{sz}{R_{\perp}^{3}}\pm\frac{(1-2\nu)s}{R_{\perp}
(R_{\perp}-z)}\right),
\end{equation}
%
where $s=x,y$ and the equation with $\pm$ correspond to $G^0_{sz}$ and $G^0_{zs}$, respectively, 
and $R_{\perp} \equiv \sqrt{x^{2} +y^{2}}$ with $x$ and $y$ the distances from the 
point force. The Young's modulus and Poisson ratio of the elastic substrate are denoted by 
$E$ and $\nu$, respectively.   
The displacement of a material point at $(x,y,z\leq 0)$ in the medium due
to a stress distribution ${\bf F}$ is simply the convolution
$\u(\r) \equiv [u_x \ u_y\  u_z]^\intercal = {\bf G^0}\Conv\F$, where 
\begin{equation}
{\bf G^0} = \left[ \begin{matrix} G^0_{xx}(x,y,z) & G^0_{xy}(x,y,z) & G^0_{xz}(x,y,z) \\
	G^0_{yx}(x,y,z) & G^0_{xy}(x,y,z) & G^0_{yz}(x,y,z) \\
	G^0_{zx}(x,y,z) & G^0_{zy}(x,y,z) & G^0_{zz}(x,y,z) 
 \end{matrix} \right]
\end{equation}


%, while for a surface-localized force distribution, 
%$\F=(F_{x}(x,y,z=0),F_{y}(x,y,z=0),F_{z}(x,y,z=0))$ the displacement would be

%\begin{equation}
%\left[\begin{array}{c} u_{x}\\u_{y}\\u_{z}\end{array}\right]=
%\left[\begin{array}{ccc} G_{xx}&G_{xy}&G_{xz}\\G_{yx}&G_{yy}&G_{yz}\\G_{zx}&G_{zy}&G_{zz}
%\end{array}\right]
%\left[\begin{array}{c} F_{x}\\F_{y}\\F_{z}\end{array}\right]
%\end{equation}
%
%
%For a force distributation
%
%\begin{equation}
%\F=(F_{x}(x,y),F_{y}(x,y),F_{z}(x,y)) 
%\end{equation}
% 
%exerting on the surface of the medium,the displacement would be:

%\begin{equation}
%u_{i}(\r) = \int \dd \r_{\perp}'\dd z'
%G_{ij}(\r_{\perp}-\r_{\perp}',z-z')
%F_{j}(\r_{\perp}', z'),
%\label{UMODEL0}
%\end{equation}
% 
%where $\r = (\r_{\perp},z)$, $\r_{\perp}=x\hat{x} + y\hat{y}$ and
%$\r_{\perp}'= x'\hat{x} + y'\hat{y}$.  We will use this expression for
%the displacement as the model for the data term in the objective
%function for our inverse problem.

For our specific problem, we shall specialize the forces to surface
stresses $\sigma_{x,y}$ that act on the plane perpendicular to the
$\hat{z}$ axis. We define the in-plane stress distribution, at depth $z$, as
$\bs(x,y,z) = \sigma_{xz}(x,y,z)\hat{x} + \sigma_{yz}(x,y,z)\hat{y}$. 
The resulting surface-level displacement fields become

\begin{align}
u_{x}(x,y) &= \int_\Omega \dd x'\dd y'G_{xx}( x-x',y-y')\sigma_{xz}(x',y') \nonumber\\
&\qquad +  \int_\Omega \dd x'\dd y'G_{xy}( x-x',y-y')\sigma_{yz}(x',y') \label{eq:UMODEL1x}  \\
u_y(x,y) &= \int_\Omega \dd x'\dd y'G_{yx}( x-x',y-y')\sigma_{xz}(x',y') \nonumber\\
&\qquad +  \int_\Omega \dd x'\dd y'G_{yy}( x-x',y-y')\sigma_{yz}(x',y'),  \label{eq:UMODEL1y}  
\end{align}
where
\begin{equation}
G_{\cdot,\cdot}(x,y) = G^0_{\cdot,\cdot}(x,y,z=0),
\end{equation}
and by abuse of notation,
\begin{equation}
\sigma_{xz}(x,y) = \sigma_{xz}(x,y,z=0) \quad \sigma_{yz}(x,y) = \sigma_{yz}(x,y,z=0) .
\end{equation}
In Eqs~\ref{eq:UMODEL1x} and~\ref{eq:UMODEL1}, we have restricted the domain of integration to the extent of the cell, $\Omega$, to emphasize that 
$\boldsymbol\sigma$ has compact support. Note that tangential stresses can lead to displacement data in the normal direction.

%




\section{Setup of inverse problem}

Here, we develop an objective function for which the minimizing
solution provides a good approximation to the underlying stress
field, while preserving discontinuities.
 The first component is simply a quadratic data mismatch term
defined by the sum over the displacements measured at the $N$
measurement positions at $\r_{i}$:


\begin{equation}
\Phi_{\rm data}[\bs] = \sum_{i}^{N}\vert \u^{\rm
  data}(\r_{i})- \u(\r_i)\vert^{2}.
\end{equation}
%
Since $\u^{\rm data}(\r_{i})$ is given, and $\u(\r_{i})$, is given by
Eqs.~\ref{eq:UMODEL1x} and~\ref{eq:UMODEL1y}, this contribution to the objective function is a
functional over the surface-stress function $\bs(\r_{\perp})$. 

We solve Eqs.~\ref{eq:UMODEL1x} and~\ref{eq:UMODEL1y} \emph{exactly} given
an approximation of the stress field.
  Let us consider the first-order approximation of $\sigma_{xz}$ and $\sigma_{yz}$ using central finite differences, for $x\in[x_j - \delta x/2, x_j+\delta x/2) \cap y\in[y_j-\delta y/2, y_j + \delta y /2)$,
\begin{align}
\lefteqn{\sigma_{xz}(x,y) =\sigma_{xz}(x_i , y_j)  }\nonumber \\
& \qquad+ (x-x_i)\frac{\sigma_{xz}(x_{i+1},y_j) -\sigma_{xz}(x_{i-1},y_j) }{2\delta x}  \nonumber\\
&\qquad + (y-y_j)\frac{\sigma_{xz}(x_i,y_{j+1}) - \sigma_{xz}(x_i,y_{j-1}) }{2\delta y} \nonumber\\
&\qquad + \mathcal{O}(\delta x)^2 + \mathcal{O}(\delta y)^2,\label{eq:sigma_affine}
\end{align}
where $i,j$ denotes a tuple of grid coordinates. The stress tensor $\boldsymbol{\sigma}$ has compact support within the region $\Omega$. For this reason, the domain of integration within Eq.~\ref{eq:ux} is compact. Noting that our approximation for the stress fields is piecewise affine, we may rewrite Eq.~\ref{eq:ux}, decomposing the integral into a sum of integrals over grid cells

\subsection{Regularization}

So far, the construction of the surface stress, even at the discrete
resolution where measurements are available, is ill-conditioned. To
regularize this problem, while obeying some physically-relevant
characteristics of the surface stress, we apply an L1 total
variational penalty onto the problem resulting in a penalized
objective function
%
\begin{equation}
\Phi = ||\sigma_{xz}||_{TV} +  ||\sigma_{yz}||_{TV}
\end{equation}
%
The corresponding optimization problem is linear and is solvable within standard optimization routines. In our implementation, we use a second-order 
quadratic cone solver 


\section{Results}

\section{summary and Conclusions}



%%%%%%%%%%%%%%%%%%%%%%%%%%%%%%%%%%%%%%%%%%%%%%%%%%%%%%%%%%%%%%%%%%%%%%%%
%%%%%%%%%%%%%%%%%%%%%%%%%%%%%%%%%%%%%%%%%%%%%%%%%%%%%%%%%%%%%%%%%%%%%%%%
\end{document}
%%%%%%%%%%%%%%%%%%%%%%%%%%%%%%%%%%%%%%%%%%%%%%%%%%%%%%%%%%%%%%%%%%%%%%%%
%%%%%%%%%%%%%%%%%%%%%%%%%%%%%%%%%%%%%%%%%%%%%%%%%%%%%%%%%%%%%%%%%%%%%%%%



where 

\begin{equation}
G_{xx} =-\frac{1+\nu}{2\pi E}\left[\frac{2(1-\nu)R_{\perp}+z}{R_{\perp}(R_{\perp}+z)} + 
\frac{[2R_{\perp}(\nu R_{\perp}+z)+z^{2}]x^{2}}{R_{\perp}^{3}(R_{\perp}+z)^{2}}\right]
\end{equation}
\begin{equation} 
G_{xy} =-\frac{1+\nu}{2\pi E}\frac{[2R_{\perp}(\nu R_{\perp}+z)+z^{2}]xy}{R_{\perp}^{3}
(R_{\perp}+z)^{2}}
\end{equation}
\begin{equation}
G_{xz} =-\frac{1+\nu}{2\pi E}\left(\frac{xz}{R_{\perp}^{3}}-\frac{(1-2\nu)x}{R_{\perp}
(R_{\perp}+z)}\right)
\end{equation}
\begin{equation}
G_{yx} =-\frac{1+\nu}{2\pi E}\frac{[2R_{\perp}(\nu R_{\perp}+z)+z^{2}]xy}{R_{\perp}^{3}(R_{\perp}
+z)^{2}}
\end{equation}
\begin{equation}
G_{yy} =-\frac{1+\nu}{2\pi E}\left[\frac{2(1-\nu)R_{\perp}+z}{R_{\perp}(R_{\perp}+z)}
+\frac{[2R_{\perp}(\nu R_{\perp}+z)+z^{2}]y^{2}}{R_{\perp}^{3}(R_{\perp}+z)^{2}}\right]
\end{equation}
\begin{equation}
G_{yz} =-\frac{1+\nu}{2\pi E}\left[\frac{yz}{R_{\perp}^{3}}-\frac{(1-2\nu)y}
{R_{\perp}(R_{\perp}+z)}\right]
\end{equation}
\begin{equation}
G_{zx} =-\frac{1+\nu}{2\pi E}\left(\frac{1-2\nu}{R_{\perp}(R_{\perp}+z)}+\frac{z}{R_{\perp}^{3}}
\right)x
\end{equation}
\begin{equation}
G_{zy} =-\frac{1+\nu}{2\pi E}\left(\frac{1-2\nu}{R_{\perp}(R_{\perp}+z)}+\frac{z}{R_{\perp}^{3}}\right)y
\end{equation}
\begin{equation}
G_{zz} =-\frac{1+\nu}{2\pi E}\left(\frac{2(1-\nu)}{R_{\perp}}+\frac{z^{2}}{R_{\perp}^{3}}\right)
\end{equation}
\begin{equation}
R_{\perp} \equiv \sqrt{x^{2} +y^{2} +z^{2}}
\end{equation}
Here,\,\,$E$\, is Young's modulus and\,\,\,$\nu$\,\,\,is Poission's ratio.
%






\begin{equation}
G_{xx} =-\frac{1+\nu}{2\pi E}\left[\frac{2(1-\nu)R_{\perp}+z}{R_{\perp}(R_{\perp}+z)} + 
\frac{[2R_{\perp}(\nu R_{\perp}+z)+z^{2}]X^{2}}{R_{\perp}^{3}(R_{\perp}+z)^{2}}\right]
\end{equation}
\begin{equation} 
G_{xy} =-\frac{1+\nu}{2\pi E}\frac{[2R_{\perp}(\nu R_{\perp}+z)+z^{2}]XY}{R_{\perp}^{3}
(R_{\perp}+z)^{2}}
\end{equation}
\begin{equation}
G_{xz} =-\frac{1+\nu}{2\pi E}\left(\frac{Xz}{R_{\perp}^{3}}-\frac{(1-2\nu)X}{R_{\perp}
(R_{\perp}+z)}\right)
\end{equation}
\begin{equation}
G_{yx} =-\frac{1+\nu}{2\pi E}\frac{[2R_{\perp}(\nu R_{\perp}+z)+z^{2}]XY}{R_{\perp}^{3}(R_{\perp}
+z)^{2}}
\end{equation}
\begin{equation}
G_{yy} =-\frac{1+\nu}{2\pi E}\left[\frac{2(1-\nu)R_{\perp}+z}{R_{\perp}(R_{\perp}+z)}
+\frac{[2R_{\perp}(\nu R_{\perp}+z)+z^{2})]Y^{2}}{R_{\perp}^{3}(R_{\perp}+z)^{2}}\right]
\end{equation}
\begin{equation}
G_{yz} =-\frac{1+\nu}{2\pi E}\left(\frac{z}{R_{\perp}^{3}}-\frac{1-2\nu}
{R_{\perp}(R_{\perp}+z)}\right)Y
\end{equation}
\begin{equation}
G_{zx} =-\frac{1+\nu}{2\pi E}\left(\frac{1-2\nu}{R_{\perp}(R_{\perp}+z)}+\frac{z}{R_{\perp}^{3}}
\right)X
\end{equation}
\begin{equation}
G_{zy} =-\frac{1+\nu}{2\pi E}\left(\frac{1-2\nu}{R_{\perp}(R_{\perp}+z)}+\frac{z}{R_{\perp}^{3}}\right)Y
\end{equation}
\begin{equation}
G_{zz} =-\frac{1+\nu}{2\pi E}\left(\frac{2(1-\nu)}{R_{\perp}}+\frac{z^{2}}{R_{\perp}^{3}}\right)
\end{equation}
\begin{equation}
R_{\perp} \equiv \sqrt{(x-x')^{2} +(y-y')^{2} +z^{2}}, \,\,\, X\equiv x-x', \,\,\, Y\equiv y-y'
\end{equation}

\noindent\section{Surface stresses only}
Particularly,if we set\,\,$z=0$,\,these coefficients become:
%
\begin{equation}
G_{xx} =-\frac{1+\nu}{2\pi E}\left[\frac{2(1-\nu)}{R_{\perp}}+ 
\frac{2\nu X^{2}}{R_{\perp}^{3}}\right]
\end{equation}
\begin{equation} 
G_{xy} =-\frac{1+\nu}{2\pi E}\frac{2\nu XY}{R_{\perp}^{3}}
\end{equation}
\begin{equation}
G_{xz} =-\frac{1+\nu}{2\pi E}\frac{2 \nu -1}{R_{\perp}^{2}}X
\end{equation}
\begin{equation}
G_{yx} =-\frac{1+\nu}{2\pi E}\frac{2\nu XY}{R_{\perp}^{3}}
\end{equation}
\begin{equation}
G_{yy} =-\frac{1+\nu}{2\pi E}\left(\frac{2(1-\nu)}{R_{\perp}}
+\frac{2\nu Y^{2}}{R_{\perp}^{3}}\right)
\end{equation}
\begin{equation}
G_{yz} =-\frac{1+\nu}{2\pi E}\frac{2 \nu -1}
{R_{\perp}^2}Y
\end{equation}
\begin{equation}
G_{zx} =-\frac{1+\nu}{2\pi E}\frac{1-2\nu}{R_{\perp}^2}
X
\end{equation}
\begin{equation}
G_{zy} =-\frac{1+\nu}{2\pi E}\frac{1-2\nu}{R_{\perp}^2}Y
\end{equation}
\begin{equation}
G_{zz} =-\frac{1+\nu}{2\pi E}\frac{2(1-\nu)}{R_{\perp}}
\end{equation}
\begin{equation}
R_{\perp} \equiv \sqrt{(x-x')^{2} +(y-y')^{2}}, \,\,\, X\equiv x-x', \,\,\, Y\equiv y-y'
\end{equation}

%
%
%

\section{Incompressible Limit} 
When the medium is considered
incompressible,ie,\,\,\,$\nu =\frac{1}{2},$\,\,\,we have simplified
results.

Throughout the medium,these coefficients would be:
\begin{equation}
G_{xx} =-\frac{3}{4\pi E}\left[\frac{1}{R_{\perp}} + 
\frac{[R_{\perp}(R_{\perp}+2z)+z^{2}]X^{2}}{R_{\perp}^{3}(R_{\perp}+z)^{2}}\right]
\end{equation}
\begin{equation} 
G_{xy} =-\frac{3}{4\pi E}\frac{[R_{\perp}(R_{\perp}+2z)+z^{2}]XY}{R_{\perp}^{3}
(R_{\perp}+z)^{2}}
\end{equation}
\begin{equation}
G_{xz} =-\frac{3}{4\pi E}\left(\frac{z}{R_{\perp}^{3}}\right)X
\end{equation}
\begin{equation}
G_{yx} =-\frac{3}{4\pi E}\frac{[R_{\perp}(R_{\perp}+2z)+z^{2}]XY}{R_{\perp}^{3}(R_{\perp}
+z)^{2}}
\end{equation}
\begin{equation}
G_{yy} =-\frac{3}{4\pi E}\left[\frac{1}{R_{\perp}}
+\frac{[R_{\perp}(R_{\perp}+2z)+z^{2}]Y^{2}}{R_{\perp}^{3}(R_{\perp}+z)^{2}}\right]
\end{equation}
\begin{equation}
G_{yz} =-\frac{3}{4\pi E}\left(\frac{z}{R_{\perp}^{3}}\right)Y
\end{equation}
\begin{equation}
G_{zx} =-\frac{3}{4\pi E}\left(\frac{z}{R_{\perp}^{3}}
\right)X
\end{equation}
\begin{equation}
G_{zy} =-\frac{3}{4\pi E}\left(\frac{z}{R_{\perp}^{3}}\right)Y
\end{equation}
\begin{equation}
G_{zz} =-\frac{3}{4\pi E}\left(\frac{1}{R_{\perp}}+\frac{z^{2}}{R_{\perp}^{3}}\right)
\end{equation}
\begin{equation}
R_{\perp} \equiv\sqrt{(x'-x)^{2} +(y'-y)^{2} +z^{2}}, \,\,\, X\equiv x-x', \,\,\, Y\equiv y-y'
\end{equation}
%
And on the surface:
\begin{equation}
G_{xx} =-\frac{3}{4\pi E}\left(\frac{1}{R_{\perp}}+ 
\frac{X^{2}}{R_{\perp}^{3}}\right)
\end{equation}
\begin{equation} 
G_{xy} =-\frac{3}{4\pi E}\frac{XY}{R_{\perp}^{3}}
\end{equation}
\begin{equation}
G_{xz} =0
\end{equation}
\begin{equation}
G_{yx} =-\frac{3}{4\pi E}\frac{XY}{R_{\perp}^{3}}
\end{equation}
\begin{equation}
G_{yy} =-\frac{3}{4\pi E}\left(\frac{1}{R_{\perp}}
+\frac{Y^{2}}{R_{\perp}^{3}}\right)
\end{equation}
\begin{equation}
G_{yz} =0
\end{equation}
\begin{equation}
G_{zx} =0
\end{equation}
\begin{equation}
G_{zy} =0
\end{equation}
\begin{equation}
G_{zz} =-\frac{3}{4\pi E}\frac{1}{R_{\perp}}
\end{equation}
\begin{equation}
R_{\perp} \equiv \sqrt{(x-x')^{2} +(y-y')^{2}}, \,\,\, X\equiv x-x', \,\,\, Y\equiv y-y'
\end{equation}




The optimization problem using TV can be formally defined by minimization of two possible 
objective functions

\begin{equation}
\Phi_{1} = \sum_{i}\bigg| \u^{\rm data}(\r_{i})- 
\u^{\rm model}(\r_i)\bigg|^{2} + \gamma \int_{\Omega}
\vert \nabla_{\perp} F_{x}(\r_{\perp})\vert^{p} \dd \r_{\perp} + \gamma \int_{\Omega}
\vert \nabla_{\perp} F_{y}(\r_{\perp})\vert^{p} \dd \r_{\perp} +
\int_{\Omega}(\lambda_{x}F_{x}+\lambda_{y}F_{y})\dd \r_{\perp},
\end{equation}
%
where $p=1,2$ and 
\begin{equation}
\Phi_{2} = \sum_{i}\bigg| u_{x}^{\rm data}(\r_{\perp}(i))- u_{x}^{\rm model}(\r_{\perp}(i))
\bigg|^{2} + \gamma \int_{\Omega}
\vert \nabla_{\perp} \cdot {\bf F}_{\perp}(\r_{\perp})\vert^{p} \dd \r_{\perp} +
\int_{\Omega}(\lambda_{x}F_{x}+\lambda_{y}F_{y})\dd \r_{\perp},
\end{equation}

Where the displacement derived from the model is

\begin{equation}
u_{x}^{\rm model}(\r_{\perp}(i)) = 
\int_{\Omega}G_{xj}(\r_{\perp}(i)-\r_{\perp}')
F_{j}(\r_{\perp}')\dd \r_{\perp}', \quad j=1,2.
\end{equation}

In this problem, $\Omega$ is the foot print of the cell, $\gamma$ is the
regularization parameter, and $\lambda$ is the Lagrange multiplier
used to enforce the total zero force condition $\int {\bf
  F}(\r_{\perp}) \dd \r_{\perp} =0$, and the points $\r_{\perp}(i)$ are
those at which data is measured.  These data points are discrete in
the $x-$direction since the surface fluorescence comes from discretely
spaces lines.  However, there can be a continuous set of data points
along the $y-$direction of the lines.

During the optimization of $\Phi$ to find the best ${\bf
  F}_{\perp}(\r_{\perp})$ is performed on the interior, under the cell
only.  The boundary condition ${\bf F}_{\perp}(\r_{\perp} \in \partial
\Omega) = 0$ is imposed.



There is yet another constraint. The total torque has to be zero since
the cell cannot exert a net torque from the air. 
So another constraint should be:


\begin{equation}
\int_{\Omega}\rp\times {\bf F}_{\perp}(\rp) \,\dd \rp = 0,
\end{equation}
%
where the origin of $\rp$ is arbitrary since 
the total force is also constrained to vanish.

I wonder if it is better to use a Helmhotz decomposition

\begin{equation}
{\bf F}_{\perp} = \nabla_{\perp} \varphi(\rp) + \nabla\times (\psi(\rp)\hat{z}).
\end{equation}
%
In this picture, the no net force constraint gives

\begin{equation}
\int_{\Omega} \nabla_{\perp} \varphi \,\dd \rp +
\int_{\Omega}\nabla_{\perp}\times(\psi\hat{z}) \,\dd \rp = 0.
\end{equation}
%
The no-torque condition then becomes

\begin{equation}
\int_{\partial \Omega} \left[(\rp\times\nabla_{\perp})\varphi - 
(\rp\cdot\nabla_{\perp})\psi(\rp)\right]\,\dd\rp = 0.
\end{equation}


 
\noindent The objective function for our problem is:
\begin{equation}
\Phi = \sum_{i}\bigg| u_{x}^{\rm data}(\r_{\perp}(i))- 
u_{x}^{\rm model}(\r_{\perp}(i))\bigg|^{2} + \gamma \int_{\Omega}
\vert \nabla_{\perp} P_{x}(\r_{\perp})\vert^{p} \dd \r_{\perp} + \gamma \int_{\Omega}
\vert \nabla_{\perp} P_{y}(\r_{\perp})\vert^{p} \dd \r_{\perp}
\end{equation}
Where\,\,$\gamma$\,\, is a regularization parameter, $\Omega$\,\,is
the area where we assume that there are no stresses outside.  Because
the data is collected on the plane\,\,$z=0$\,\,,and the substrate is
incompressible, i.e., $\nu=0.5$, plug these two conditions into
Equation(5) and (8), we have\,\,$G_{xz}=G_{yz}=0$\,\,, which means
that the gravity of the cells have no effect on the strain in x- and
y- direction on the surface of the substrate. When\,\,$p=1$, if we
use the anisotropic l1-norm of the derivatives for the TV norm
instead, the objective function would be:
\begin{equation}
\begin{split}
\Phi = \sum_{i}\bigg| u_{x}^{\rm data}(\r_{\perp}(i))- 
u_{x}^{\rm model}(\r_{\perp}(i))\bigg|^{2} + \gamma \int_{\Omega}
\mid \frac{\partial P_{x}(\r_{\perp})}{\partial x}\mid \dd \r_{\perp} + \gamma \int_{\Omega}
\mid \frac{\partial P_{x}(\r_{\perp})}{\partial y}\mid \dd \r_{\perp}\\+\gamma \int_{\Omega}
\mid \frac{\partial P_{y}(\r_{\perp})}{\partial x}\mid \dd \r_{\perp} + \gamma \int_{\Omega}
\mid \frac{\partial P_{y}(\r_{\perp})}{\partial y}\mid \dd \r_{\perp}
\end{split}
\end{equation}
Both the isotropic one and the anisotropic one are used in image
processing. Another consideration is that the total force and the
total torque are nearly zero. Which are equivalent to the constraints:
\begin{equation}
\int_{\Omega}\F\dd \r_{\perp}=0\,\,\,\int_{\Omega}\r_{\perp} \times \F\dd\r_{\perp}=0
\end{equation}
